\chapter{Arhitektura i dizajn sustava}
		
		\textbf{\textit{dio 1. revizije}}\\

		\textit{ Potrebno je opisati stil arhitekture te identificirati: podsustave, preslikavanje na radnu platformu, spremišta podataka, mrežne protokole, globalni upravljački tok i sklopovsko-programske zahtjeve. Po točkama razraditi i popratiti odgovarajućim skicama:}
	\begin{itemize}
		\item 	\textit{izbor arhitekture temeljem principa oblikovanja pokazanih na predavanjima (objasniti zašto ste baš odabrali takvu arhitekturu)}
		\item 	\textit{organizaciju sustava s najviše razine apstrakcije (npr. klijent-poslužitelj, baza podataka, datotečni sustav, grafičko sučelje)}
		\item 	\textit{organizaciju aplikacije (npr. slojevi frontend i backend, MVC arhitektura) }		
	\end{itemize}
		
		\text	Arhitektura našeg projekta podijeljena je na tri generalna dijela:
		\begin{packed_item}
			\item yuh
			\item yuh
			\item yuh
		\end{packed_item}

		

				
		\section{Baza podataka}
			
			\text Za potrebe našeg projekta koristili smo relacijsku bazu podataka napisanu u programskom jeziku Java, pokrenutu \textit{engine-om} H2. Prednost te vrste baze podataka jest jednostavnost modeliranja pravog svijeta na temelju relacija. Zadaća baze podataka u projektu jest pohrana, izmjena i dohvat podataka koji se zatim obrađuju. Baza podataka sastoji se od entiteta:
			\begin{packed_item}
				\item Roditelj
				\item Dijete
				\item Pedijatar
				\item Liječnik obiteljske medicine
				\item Medicinski karton
				\item Pregled
				\item Nalaz
			\end{packed_item}
		
			\subsection{Opis tablica}
			

				\textbf{Roditelj} - Entitet sadržava sve podatke koje roditelj unosi prilikom registracije. Oni su: OIB, ime i prezime, datum rođenja, korisničko ime, lozinka, broj telefona, e-mail, poštanski broj, mjesto prebivališta, e-mail poslodavca te šifra liječnika. Entitet roditelj u odnosu je \textit{Many-to-One} s entitetom Liječnik obiteljske medicine preko šifre tog liječnika, u odnosu \textit{One-to-Many} sa entitetom Dijete, putem vlastitog OIB-a, te u odnosu \textit{One-to-One} s entitetom Medicinski karton, putem vlastitog OIB-a.
				
				
				\begin{longtblr}[
					label=none,
					entry=none
					]{
						width = \textwidth,
						colspec={|X[10,l]|X[6, l]|X[20, l]|}, 
						rowhead = 1,
					} %definicija širine tablice, širine stupaca, poravnanje i broja redaka naslova tablice
					\hline \SetCell[c=3]{c}{\textbf{Roditelj}}	 \\ \hline[3pt]
					\SetCell{LightGreen}OIB & CHAR(11)	&  	Jedinstveni identifikator roditelja	\\ \hline
					nameParent	& VARCHAR &   Ime roditelja	\\ \hline 
					lastNameParent	& VARCHAR &   Prezime roditelja	\\ \hline 
					dateOfBirthParent	& DATE &  Datum rođenja roditelja 	\\ \hline 
					userNameParent	& VARCHAR &   Korisničko ime računa	\\ \hline 
					passwordParent	& VARCHAR &   Zaporka računa	\\ \hline 
					phoneNumberParent	& VARCHAR &   Broj telefona roditelja	\\ \hline 
					emailParent	& VARCHAR &   E-mail roditelja	\\ \hline 
					postalCode & INT &  Poštanski broj mjesta stanovanja \\ \hline 
					placeOfResidence & VARCHAR	&  	Mjesto stanovanja roditelja	\\ \hline 
					employerEmail	& VARCHAR &   E-mail poslodavca roditelja	\\ \hline 
					\SetCell{LightBlue}doctorId	& INT &   Identifikator L.O.M.-a roditelja	\\ \hline
				\end{longtblr}
				
				\textbf{Dijete} - Entitet sadržava sve podatke koje unosi pedijatar prilikom prvog pregleda djeteta. Oni su: OIB, OIB roditelja, šifra pedijatra, ime i prezime, datum rođenja, naziv škole/vrtića te e-mail škole/vrtića. Entitet dijete u odnosu je \textit{Many-to-One} sa entitetom Roditelj, putem roditeljevog OIB-a, u odnosu \textit{Many-to-One} s entitetom Pedijatar putem pedijatrovog ID-a, te u odnosu \textit{One-to-One} s entitetom Medicinski karton, putem vlastitog OIB-a.
				
				\begin{longtblr}[
					label=none,
					entry=none
					]{
						width = \textwidth,
						colspec={|X[11,l]|X[6, l]|X[20, l]|}, 
						rowhead = 1,
					} %definicija širine tablice, širine stupaca, poravnanje i broja redaka naslova tablice
					\hline \SetCell[c=3]{c}{\textbf{Dijete}}	 \\ \hline[3pt]
					\SetCell{LightGreen}OIB & CHAR(11)	&  	Jedinstveni identifikator djeteta 	\\ \hline
					nameChild	& VARCHAR &   Ime djeteta	\\ \hline 
					lastNameChild	& VARCHAR &   Prezime djeteta	\\ \hline 
					dateOfBirthChild	& DATE &   Datum rođenja djeteta	\\ \hline 
					educationalInstitution	& VARCHAR &   Ime škole/vrtića djeteta	\\ \hline 
					emailEduInstitution & VARCHAR &  E-mail škole/vrtića \\ \hline 
					\SetCell{LightBlue}parentOib & CHAR(11)	&  Jedinstveni identifikator roditelja	\\ \hline 
					\SetCell{LightBlue} pediatricanId	& INT &   Jedinstveni identifikator pedijatra	\\ \hline 
				\end{longtblr}
				
				\textbf{Pedijatar} - Entitet sadržava sve podatke koje unosi administrator prilikom registracije pedijatra u sustav. Oni su: šifra pedijatra, ime i prezime, datum rođenja, korisničko ime, lozinka, broj telefona, e-mail. Entitet pedijatar u odnosu je \textit{One-to-Many} s entitetom Dijete putem vlastite šifre.
				
				\begin{longtblr}[
					label=none,
					entry=none
					]{
						width = \textwidth,
						colspec={|X[12,l]|X[6, l]|X[20, l]|}, 
						rowhead = 1,
					} %definicija širine tablice, širine stupaca, poravnanje i broja redaka naslova tablice
					\hline \SetCell[c=3]{c}{\textbf{Pedijatar}}	 \\ \hline[3pt]
					\SetCell{LightGreen}pediatricianId & INT	&  	Jedinstveni identifikator pedijatra	\\ \hline
					namePediatirican	& VARCHAR &   Ime pedijatra	\\ \hline 
					lastNamePediatirican	& VARCHAR &   Prezime pedijatra	\\ \hline 
					dateOfBirthPediatirican	& DATE &   Datum rođenja pedijatra	\\ \hline 
					userNamePediatirican	& VARCHAR &   Korisničko ime računa	\\ \hline 
					passwordPediatrician & VARCHAR &  Zaporka računa \\ \hline 
					phoneNumPediatrician	& VARCHAR &   Broj telefona pedijatra	\\ \hline 
					emailPediatrician	& VARCHAR &   E-mail pedijatra	\\ \hline
					
				\end{longtblr}
				
				\textbf{Liječnik obiteljske medicine} - Entitet sadržava sve podatke koje unosi administrator prilikom registracije LOM-a u sustav. Oni su: šifra doktora, ime i prezime, datum rođenja, korisničko ime, lozinka, broj telefona i e-mail. Entitet Liječnik obiteljske medicine u odnosu je \textit{One-to-Many} s entitetom Roditelj putem vlastite šifre.
				
				\begin{longtblr}[
					label=none,
					entry=none
					]{
						width = \textwidth,
						colspec={|X[10,l]|X[6, l]|X[20, l]|}, 
						rowhead = 1,
					} %definicija širine tablice, širine stupaca, poravnanje i broja redaka naslova tablice
					\hline \SetCell[c=3]{c}{\textbf{Liječnik obiteljske medicine}}	 \\ \hline[3pt]
					\SetCell{LightGreen}doctorId & INT	&  	Jedinstveni identifikator liječnika	\\ \hline
					nameDoctor	& VARCHAR &   Ime liječnika	\\ \hline 
					lastNameDoctor	& VARCHAR &   Prezime liječnika	\\ \hline 
					dateOfBirthDoctor	& DATE &   Datum rođenja liječnika	\\ \hline 
					userNameDoctor	& VARCHAR &   Korisničko ime računa	\\ \hline 
					passwordDoctor & VARCHAR &  Zaporka računa \\ \hline 
					phoneNumDoctor	& VARCHAR &   Broj telefona liječnika	\\ \hline 
					emailDoctor	& VARCHAR &   E-mail liječnika	\\ \hline
					
				\end{longtblr}
				
				\textbf{Medicinski karton} - Entitet sadržava podatke za opis medicinskog kartona pojedinog pacijenta (roditelj/dijete). Oni su: šifra kartona, OIB, trenutna dijagnoza i popis alergija. Entitet Medicinski karton u odnosu je \textit{One-to-One} s entitetom Roditelj i s entitetom Dijete, putem OIB-a pacijenta (R/D), te u odnosima \textit{One-to-Many} s entitetima Nalaz i Pregled.  
				
				\begin{longtblr}[
					label=none,
					entry=none
					]{
						width = \textwidth,
						colspec={|X[8,l]|X[6, l]|X[20, l]|}, 
						rowhead = 1,
					} %definicija širine tablice, širine stupaca, poravnanje i broja redaka naslova tablice
					\hline \SetCell[c=3]{c}{\textbf{Medicinski karton}}	 \\ \hline[3pt]
					\SetCell{LightGreen}redordId & INT	&  	Jedinstveni identifikator kartona	\\ \hline
					\SetCell{LightBlue}OIB	& CHAR(11) &  Jedinstveni identifikator pacijenta 	\\ \hline 
					currentDiagnosis & VARCHAR &  Aktivna ili zadnja dijagnoza djeteta \\ \hline 
					allergyList & VARCHAR	&  	Popis alergija djeteta	\\ \hline
				\end{longtblr}
				
				\textbf{Pregled} - Entitet sadržava podatke pregleda kojeg održava pedijatar ili LOM. Oni su: šifra pregleda, šifra kartona kojem pripada, dijagnoza i datum pregleda. Entitet Pregled u odnosu je \textit{Many-to-One} s entitetom Medicinski karton.
				
				\begin{longtblr}[
					label=none,
					entry=none
					]{
						width = \textwidth,
						colspec={|X[9,l]|X[6, l]|X[20, l]|}, 
						rowhead = 1,
					} %definicija širine tablice, širine stupaca, poravnanje i broja redaka naslova tablice
					\hline \SetCell[c=3]{c}{\textbf{Pregled}}	 \\ \hline[3pt]
					\SetCell{LightGreen}examinationId & INT	&  	Jedinstveni identifikator pregleda	\\ \hline
					\SetCell{LightBlue}recordId	& INT &  Jedinstveni identifikator kartona 	\\ \hline 
					diagnosis & VARCHAR &  Opis dijagnoze  \\ \hline 
					dateOfExamination & DATE	&  	Datum pregleda	\\ \hline
				\end{longtblr}
				
				\textbf{Nalaz} - Entitet sadržava podatke koji opisuju nalaz dobiven prilikom privatnog pregleda, kojeg \textit{uploada} roditelj. Oni su: šifra nalaza, šifra kartona, datum nalaza, podaci nalaza. Entitet Nalaz u odnosu je \textit{Many-to-One} s entitetom Medicinski karton.
				
				\begin{longtblr}[
					label=none,
					entry=none
					]{
						width = \textwidth,
						colspec={|X[9,l]|X[6, l]|X[20, l]|}, 
						rowhead = 1,
					} %definicija širine tablice, širine stupaca, poravnanje i broja redaka naslova tablice
					\hline \SetCell[c=3]{c}{\textbf{Nalaz}}	 \\ \hline[3pt]
					\SetCell{LightGreen}reportId & INT	&  	Jedinstveni identifikator nalaza	\\ \hline
					\SetCell{LightBlue}recordId	& INT &  Jedinstveni identifikator kartona 	\\ \hline 
					reportInformation & VARCHAR &  Podaci iz nalaza  \\ \hline 
					dateOfReport & DATE	&  	Datum pregleda	\\ \hline
				\end{longtblr}
				
			\subsection{Dijagram baze podataka}
				\textit{ U ovom potpoglavlju potrebno je umetnuti dijagram baze podataka. Primarni i strani ključevi moraju biti označeni, a tablice povezane. Bazu podataka je potrebno normalizirati. Podsjetite se kolegija "Baze podataka".}
			
			\eject
			
			
		\section{Dijagram razreda}
		
			\textit{Potrebno je priložiti dijagram razreda s pripadajućim opisom. Zbog preglednosti je moguće dijagram razlomiti na više njih, ali moraju biti grupirani prema sličnim razinama apstrakcije i srodnim funkcionalnostima.}\\
			
			\textbf{\textit{dio 1. revizije}}\\
			
			\textit{Prilikom prve predaje projekta, potrebno je priložiti potpuno razrađen dijagram razreda vezan uz \textbf{generičku funkcionalnost} sustava. Ostale funkcionalnosti trebaju biti idejno razrađene u dijagramu sa sljedećim komponentama: nazivi razreda, nazivi metoda i vrste pristupa metodama (npr. javni, zaštićeni), nazivi atributa razreda, veze i odnosi između razreda.}\\
			
			\textbf{\textit{dio 2. revizije}}\\			
			
			\textit{Prilikom druge predaje projekta dijagram razreda i opisi moraju odgovarati stvarnom stanju implementacije}
			
			
			
			\eject
		
		\section{Dijagram stanja}
			
			
			\textbf{\textit{dio 2. revizije}}\\
			
			\textit{Potrebno je priložiti dijagram stanja i opisati ga. Dovoljan je jedan dijagram stanja koji prikazuje \textbf{značajan dio funkcionalnosti} sustava. Na primjer, stanja korisničkog sučelja i tijek korištenja neke ključne funkcionalnosti jesu značajan dio sustava, a registracija i prijava nisu. }
			
			
			\eject 
		
		\section{Dijagram aktivnosti}
			
			\textbf{\textit{dio 2. revizije}}\\
			
			 \textit{Potrebno je priložiti dijagram aktivnosti s pripadajućim opisom. Dijagram aktivnosti treba prikazivati značajan dio sustava.}
			
			\eject
		\section{Dijagram komponenti}
		
			\textbf{\textit{dio 2. revizije}}\\
		
			 \textit{Potrebno je priložiti dijagram komponenti s pripadajućim opisom. Dijagram komponenti treba prikazivati strukturu cijele aplikacije.}