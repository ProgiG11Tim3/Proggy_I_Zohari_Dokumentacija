\chapter{Opis projektnog zadatka}
		
		\text Cilj ovog projekta razviti je programsku podršku za web aplikaciju \textit{Ozdravi} koja će korisnicima omogućiti olakšanu komunikaciju s pedijatrom i liječnikom obiteljske medicine, te lakši pregled podataka o pregledima sebe i svojeg djeteta. Uz to, aplikacija će automatizirati slanje preporuka za bolovanje i ispričnica, uvelike štedeći vrijeme roditeljima koji zbog toga neće morati naknadno ići po imenovane potvrde.\\
		Prilikom pokretanja aplikacije neprijavljenom korisniku prikazat će se naslovna web stranica s opisom funkcionalnosti aplikacije, katalogom usluga, te opcijama za registraciju ili prijavu. \\
		Prilikom registracije korisnik unosi sljedeće podatke:
		\begin{packed_item}
			
			\item  korisničko ime
			\item  email adresa
			\item  lozinka
			\item  ime
			\item  prezime
			\item  OIB
		\end{packed_item}
		Registracijom u sustav, korisnik dobiva prava roditelja. Naknadna promjena prava je nemoguća. Ostale uloge se ne registriraju, već su dodane u sustav od strane administratora. 
		
		\underbar{\textit{Roditelj}} prijavom u sustav koristeći svoje korisničko ime i lozinku dolazi do uvodne stranice za roditelje. Na uvodnoj stranici nalaze se obavijesti od liječnika obiteljske medicine i pedijatra, te izbornik mogućih profila kojima roditelj može pristupiti. Roditelj ima omogućen pristup svojem profilu i profilima svoje djece. Nakon biranja profila, otvara se nova stranica na kojoj se nalazi izbornik mogućnosti, te prostor za prikaz. \\
		Unutar profila djeteta, izbornik sadrži sljedeće opcije:
		\begin{packed_item}
			
			\item  Obavijesti - otvara prikaz svih obavijesti od pedijatra zaduženog za navedeno dijete
			\item  Povijest liječničkih pregleda - otvara svojevrstan medicinski karton djeteta
			\item  Generirane ispričnice - otvara pregled generiranih ispričnica koje pedijatar izdaje djetetu
			\item  Nalazi iz laboratorija - otvara pregled nalaza djeteta koji su naknadno dobiveni iz laboratorija
			\item  Specijalistički pregledi - otvara stranicu na kojoj pedijatar šalje potvrdu za specijalistički pregled i lokacije na kojima je moguće izvršiti navedeni pregled
			\item  Učitavanje nalaza - opcija koja omogućuje \textit{upload} nalaza koji je dobiven pri eventualnom pregledu kod privatnika
		
		\end{packed_item}
		Unutar profila roditelja, izbornik sadrži sljedeće opcije:
			\begin{packed_item}
			
			\item  Obavijesti - otvara prikaz svih obavijesti od liječnika zaduženog za roditelja
			\item  Povijest liječničkih pregleda - otvara svojevrstan medicinski karton roditelja
			\item  Potvrđena bolovanja - otvara pregled bolovanja koja su odobrena od strane liječnika
			\item  Specijalistički pregledi - otvara stranicu na kojoj liječnik šalje potvrdu za specijalistički pregled i lokacije na kojima je moguće izvršiti navedeni pregled
			\item  Učitavanje nalaza - opcija koja omogućuje \textit{upload} nalaza koji je dobiven pri eventualnom pregledu kod privatnika
			
		\end{packed_item}
		
		\noindent Osim roditelja, postoje još tri vrste korisnika:
		
		\begin{packed_item}
			
			\item  pedijatar
			\item  liječnik obiteljske medicine
			\item  administrator
			
		\end{packed_item}	
		
		\underbar{\textit{Pedijatar}} prijavom u sustav ulazi na stranicu na kojoj se nalazi izbornik 
		
		\section{Primjeri u \LaTeX u}
		
		\textit{Ovo potpoglavlje izbrisati.}\\

		U nastavku se nalaze različiti primjeri kako koristiti osnovne funkcionalnosti \LaTeX a koje su potrebne za izradu dokumentacije. Za dodatnu pomoć obratiti se asistentu na projektu ili potražiti upute na sljedećim web sjedištima:
		\begin{itemize}
			\item Upute za izradu diplomskog rada u \LaTeX u - \url{https://www.fer.unizg.hr/_download/repository/LaTeX-upute.pdf}
			\item \LaTeX\ projekt - \url{https://www.latex-project.org/help/}
			\item StackExchange za Tex - \url{https://tex.stackexchange.com/}\\
		
		\end{itemize} 	


		
		\noindent \underbar{podcrtani tekst}, \textbf{podebljani tekst}, 	\textit{nagnuti tekst}\\
		\noindent \normalsize primjer \large primjer \Large primjer \LARGE {primjer} \huge {primjer} \Huge primjer \normalsize
				
		\begin{packed_item}
			
			\item  primjer
			\item  primjer
			\item  primjer
			\item[] \begin{packed_enum}
				\item primjer
				\item[] \begin{packed_enum}
					\item[1.a] primjer
					\item[b] primjer
				\end{packed_enum}
				\item primjer
			\end{packed_enum}
			
		\end{packed_item}
		
		\noindent primjer url-a: \url{https://www.fer.unizg.hr/predmet/proinz/projekt}
		
		\noindent posebni znakovi: \# \$ \% \& \{ \} \_ 
		$|$ $<$ $>$ 
		\^{} 
		\~{} 
		$\backslash$ 
		
		
		\begin{longtblr}[
			label=none,
			entry=none
			]{
				width = \textwidth,
				colspec={|X[8,l]|X[8, l]|X[16, l]|}, 
				rowhead = 1,
			} %definicija širine tablice, širine stupaca, poravnanje i broja redaka naslova tablice
			\hline \SetCell[c=3]{c}{\textbf{naslov unutar tablice}}	 \\ \hline[3pt]
			\SetCell{LightGreen}IDKorisnik & INT	&  	Lorem ipsum dolor sit amet, consectetur adipiscing elit, sed do eiusmod  	\\ \hline
			korisnickoIme	& VARCHAR &   	\\ \hline 
			email & VARCHAR &   \\ \hline 
			ime & VARCHAR	&  		\\ \hline 
			\SetCell{LightBlue} primjer	& VARCHAR &   	\\ \hline 
		\end{longtblr}
		

		\begin{longtblr}[
				caption = {Naslov s referencom izvan tablice},
				entry = {Short Caption},
			]{
				width = \textwidth, 
				colspec = {|X[8,l]|X[8,l]|X[16,l]|}, 
				rowhead = 1,
			}
			\hline
			\SetCell{LightGreen}IDKorisnik & INT	&  	Lorem ipsum dolor sit amet, consectetur adipiscing elit, sed do eiusmod  	\\ \hline
			korisnickoIme	& VARCHAR &   	\\ \hline 
			email & VARCHAR &   \\ \hline 
			ime & VARCHAR	&  		\\ \hline 
			\SetCell{LightBlue} primjer	& VARCHAR &   	\\ \hline 
		\end{longtblr}
	


		
		
		%unos slike
		\begin{figure}[H]
			\includegraphics[scale=0.4]{slike/aktivnost.PNG} %veličina slike u odnosu na originalnu datoteku i pozicija slike
			\centering
			\caption{Primjer slike s potpisom}
			\label{fig:promjene}
		\end{figure}
		
		\begin{figure}[H]
			\includegraphics[width=\textwidth]{slike/aktivnost.PNG} %veličina u odnosu na širinu linije
			\caption{Primjer slike s potpisom 2}
			\label{fig:promjene2} %label mora biti drugaciji za svaku sliku
		\end{figure}
		
		Referenciranje slike \ref{fig:promjene2} u tekstu.
		
		\eject
		
	