\chapter{Specifikacija programske potpore}
		
	\section{Funkcionalni zahtjevi}
			\noindent \textbf{Dionici:}
			
			\begin{packed_enum}
				
				\item Vlasnik (naručitelj)
				\item Roditelji				
				\item Liječnici i pedijatri
				\item Administrator
				\item Razvojni tim
				
			\end{packed_enum}
			
			\noindent \textbf{Aktori i njihovi funkcionalni zahtjevi:}
			
			
			\begin{packed_enum}
				\item  \underbar{Neregistrirani/neprijavljeni korisnik (inicijator) može:}
				
				\begin{packed_enum}
					
					\item pregledati katalog usluga aplikacije
					\item vidjeti koje sve vrste korisnika aplikacija podržava
					\item registrirati se u sustav, koristeći korisničko ime, email adresu, lozinku, ime, prezime i OIB, čime stvara osobni korisnički račun
					\item prijaviti se u sustav putem korisničkog imena i lozinke
					
				\end{packed_enum}
			
				\item  \underbar{Roditelj (inicijator) može:}
				
				\begin{packed_enum}
					
					\item pristupiti profilima svoje djece ili svojem profilu
					\item otvarati i čitati obavijesti vezane uz svaki profil, poslane od liječnika ili pedijatra
					\item vidjeti svoj ili djetetov medicinski karton te povijest pregleda
					\item pristupiti ispričnicama generiranim za pojedino dijete te potvrdama o bolovanju
					\item pristupiti naknadnim nalazima laboratorijskih pretraga
					\item primiti informacije i narudžbe na specijalističke preglede, te vidjeti lokacije na kojima je moguće izvesti navedeni pregled
					\item učitati (\textit{uploadati}) nalaze koje dobije prilikom pregleda u privatnoj ordinaciji
					
				\end{packed_enum}
				
				\item  \underbar{Pedijatar (inicijator) može:}
				
				\begin{packed_enum}
					
					\item pristupiti profilima djece kojima je dedicirani pedijatar
					\item pristupiti liječničkim kartonima svih svojih pacijenata
					\item prijaviti novo dijete prilikom prvog pregleda, koristeći osobne podatke djeteta (ime, prezime, OIB, datum rođenja), te OIB roditelja
					\item za svakog pacijenta unijeti novi pregled
					\item za svakog pacijenta generirati ispričnicu
					\item za roditelje djece preporučiti bolovanje
					\item naručiti dijete na specijalistički pregled, te preporučiti lokacije za izvedbu istog
					
				\end{packed_enum}
				
				\item  \underbar{Liječnik obiteljske medicine(inicijator) može:}
				
				\begin{packed_enum}
					
					\item pristupiti profilima roditelja kojima je liječnik
					\item pristupiti liječničkim kartonima svih svojih pacijenata
					\item za svakog pacijenta unijeti novi pregled
					\item potvrditi ili odbiti preporuke za bolovanje roditelja
					\item naručiti roditelja na specijalistički pregled, te preporučiti lokacije za izvedbu istog
					
				\end{packed_enum}
				
				\item  \underbar{Administrator(inicijator) može:}
				
				\begin{packed_enum}
					
					\item vidjeti popis svih registriranih korisnika i njihovih osobnih podataka
					\item brisati korisnike
					\item mijenjati veze između korisnika, npr. premjestiti dijete s profila jednog roditelja na profil drugog
					\item stvarati profile liječnika i pedijatara
					
				\end{packed_enum}
				
				\item  \underbar{Baza podataka(sudionik):}
				
				\begin{packed_enum}
					
					\item pohranjuje sve podatke o korisnicima, njihove međusobne povezanosti i uloge
					\item pohranjuje liječničke kartone i dijagnoze roditelja i djece
					
				\end{packed_enum}
				
			\end{packed_enum}
			
			\eject 
			
			
				
			\subsection{Obrasci uporabe}
					\noindent \underbar{\textbf{UC1 - Pregled usluga aplikacije}}
					\begin{packed_item}
	
						\item \textbf{Glavni sudionik: }Neprijavljeni korisnik
						\item  \textbf{Cilj:} Upoznati se s mogućnostima aplikacije
						\item  \textbf{Sudionici:}-
						\item  \textbf{Preduvjet:}-
						\item  \textbf{Opis osnovnog tijeka:}
						
						\item[] \begin{packed_enum}
	
							\item Korisnik otvara web stranicu aplikacije
							\item Korisnik može \textit{scrollati} po stranici, čitajući usluge
						\end{packed_enum}
					\end{packed_item}
					
					\noindent \underbar{\textbf{UC2-Registracija}}
					\begin{packed_item}
						
						\item \textbf{Glavni sudionik: }Korisnik
						\item  \textbf{Cilj:} Stvoriti korisnički račun roditelja za prijavu u sustav
						\item  \textbf{Sudionici:} Baza podataka
						\item  \textbf{Preduvjet:} -
						\item  \textbf{Opis osnovnog tijeka:}
						
						\item[] \begin{packed_enum}
							
							\item Na početnoj stranici aplikacije korisnik odabire opciju "Registriraj se"
							\item Korisnik unosi osobne i korisničke podatke, te potvrđuje da se želi registrirati
							\item Korisnik prima vizualnu obavijest o registraciji
						\end{packed_enum}
						
						\item  \textbf{Opis mogućih odstupanja:}
						
						\item[] \begin{packed_item}
							
							\item[2.a] Odabir već korištenog korisničkog imena, emaila ili OIB-a/nepravilan format unosa
							\item[] \begin{packed_enum}
								
								\item Korisnika se obavještava o neuspjehu registracije, i vraća ga se na stranicu za registraciju
								\item Korisnik ispravlja nepravilno unesene podatke
								
							\end{packed_enum}
						\end{packed_item}
					\end{packed_item}
					
					\noindent \underbar{\textbf{UC3 - Prijava u sustav}}
					\begin{packed_item}
						
						\item \textbf{Glavni sudionik: }Roditelj/Pedijatar/Liječnik obiteljske medicine
						\item  \textbf{Cilj:} Prijava u sustav čime se pristupa korisničkom profilu
						\item  \textbf{Sudionici:} Baza podataka
						\item  \textbf{Preduvjet:} Registracija roditelja, postojanje profila pedijatra i liječnika u sustavu
						\item  \textbf{Opis osnovnog tijeka:}
						
						\item[] \begin{packed_enum}
							
							\item Korisnik odabire opciju "Prijavi se" na početnoj stranici aplikacije
							\item Korisnik unosi svoje korisničko ime i lozinku
							\item Nakon provjere unesenih podataka u bazi podataka, korisniku se otvara stranica profila
						\end{packed_enum}
						
						\item  \textbf{Opis mogućih odstupanja:}
						
						\item[] \begin{packed_item}
							
							\item[2.a] Nepravilan unos imena ili lozinke
							\item[] \begin{packed_enum}
								
								\item Korisnik dobiva informaciju o neuspjeloj prijavi, i vraća ga se na stranicu za prijavu
								\item Korisnik ispravlja nepravilno unesene podatke
								
							\end{packed_enum}	
						\end{packed_item}
					\end{packed_item}
					
					\noindent \underbar{\textbf{UC4 - Pregled obavijesti (1)}}
					\begin{packed_item}
						
						\item \textbf{Glavni sudionik: }Roditelj
						\item  \textbf{Cilj:} Informiranje roditelja o svim novostima vezanim uz profile njih i djece
						\item  \textbf{Sudionici:} Baza podataka
						\item  \textbf{Preduvjet:} Prijava roditelja u sustav
						\item  \textbf{Opis osnovnog tijeka:}
						
						\item[] \begin{packed_enum}
							
							\item Roditelj čita popis obavijesti koji mu se prikazuje na lijevoj strani sučelja nakon uspješne prijave
							
						\end{packed_enum}
					\end{packed_item}
					
					\noindent \underbar{\textbf{UC5 - Odabir profila}}
					\begin{packed_item}
						
						\item \textbf{Glavni sudionik: }Roditelj
						\item  \textbf{Cilj:} Pristup podacima i funkcijama pojedinog profila roditelja ili djeteta
						\item  \textbf{Sudionici:} Baza podataka
						\item  \textbf{Preduvjet:} Prijava roditelja u sustav
						\item  \textbf{Opis osnovnog tijeka:}
						
						\item[] \begin{packed_enum}
							
							\item Roditelj odabire jedan od profila koji mu se prikazuju na desnoj strani sučelja nakon uspješne prijave
							
						\end{packed_enum}
					\end{packed_item}
					
					\noindent \underbar{\textbf{UC6 - Pregled obavijesti (2)}}
					\begin{packed_item}
						
						\item \textbf{Glavni sudionik: }Roditelj
						\item  \textbf{Cilj:} Informiranje roditelja o novostima vezanim uz odabrani profil
						\item  \textbf{Sudionici:} Baza podataka
						\item  \textbf{Preduvjet:} Odabir profila djeteta ili roditelja
						\item  \textbf{Opis osnovnog tijeka:}
						
						\item[] \begin{packed_enum}
							
							\item Roditelj čita popis obavijesti koji mu se prikazuje na desnoj strani sučelja nakon uspješne odabira profila \\

						\end{packed_enum}
					\end{packed_item}
					
					\noindent \underbar{\textbf{UC7 - Pregled povijesti liječničkih pregleda}}
					\begin{packed_item}
						
						\item \textbf{Glavni sudionik: }Roditelj
						\item  \textbf{Cilj:} Pregled povijesti liječničkih pregleda osobe čijem je profilu roditelj pristupio
						\item  \textbf{Sudionici:} Baza podataka
						\item  \textbf{Preduvjet:} Odabir profila djeteta ili roditelja
						\item  \textbf{Opis osnovnog tijeka:}
						
						\item[] \begin{packed_enum}
							
							\item Na izborniku s lijeve strane sučelja roditelj odabire opciju "Povijest liječničkih pregleda"
							\item Otvara se prikaz povijesti liječničkih pregleda na desnoj strani sučelja pored izbornika
						\end{packed_enum}
					\end{packed_item}
					
					\noindent \underbar{\textbf{UC8 - Pregled generiranih ispričnica}}
					\begin{packed_item}
						
						\item \textbf{Glavni sudionik: }Roditelj
						\item  \textbf{Cilj:} Pregled generiranih ispričnica za dijete čijem je profilu roditelj pristupio
						\item  \textbf{Sudionici:} Baza podataka
						\item  \textbf{Preduvjet:} Odabir profila djeteta
						\item  \textbf{Opis osnovnog tijeka:}
						
						\item[] \begin{packed_enum}
							
							\item Na izborniku s lijeve strane sučelja roditelj odabire opciju "Generirane ispričnice"
							\item Otvara se prikaz generiranih ispričnica na desnoj strani sučelja pored izbornika
						\end{packed_enum}
					\end{packed_item}
					
					\noindent \underbar{\textbf{UC9 - Pregled laboratorijskih nalaza}}
					\begin{packed_item}
						
						\item \textbf{Glavni sudionik: }Roditelj
						\item  \textbf{Cilj:} Pregled laboratorijskih nalaza za dijete čijem je profilu roditelj pristupio
						\item  \textbf{Sudionici:} Baza podataka
						\item  \textbf{Preduvjet:} Odabir profila djeteta ili roditelja
						\item  \textbf{Opis osnovnog tijeka:}
						
						\item[] \begin{packed_enum}
							
							\item Na izborniku s lijeve strane sučelja roditelj odabire opciju "Nalazi iz laboratorija"
							\item Otvara se popis laboratorijskih nalaza, na desnoj strani sučelja pored izbornika
							\item Roditelj može odabrati jedan, koji se zatim preuzima iz baze podataka na uređaj roditelja
						\end{packed_enum}
					\end{packed_item}
					
					\noindent \underbar{\textbf{UC10 - Specijalistički pregled}}
					\begin{packed_item}
						
						\item \textbf{Glavni sudionik: }Roditelj
						\item  \textbf{Cilj:} Prikaz narudžbe na specijalističkei pregledi i lokacija za isti
						\item  \textbf{Sudionici:} Baza podataka
						\item  \textbf{Preduvjet:} Odabir profila djeteta ili roditelja
						\item  \textbf{Opis osnovnog tijeka:}
						
						\item[] \begin{packed_enum}
							
							\item Na izborniku s lijeve strane sučelja roditelj odabire opciju "Specijalistički pregledi"
							\item Otvara se prikaz popisa specijalističkih pregleda koji se moraju izvršiti, na desnoj strani sučelja pored izbornika
							\item Roditelj odabire jedan
							\item Otvara se prikaz s nazivom pregleda, uputama od pedijatra ili liječnika te karta s lokacijama na kojima je moguće izvesti imenovani pregled 
						\end{packed_enum}

					\end{packed_item}
					
					\noindent \underbar{\textbf{UC11 - Učitavanje nalaza}}
					\begin{packed_item}
						
						\item \textbf{Glavni sudionik: }Roditelj
						\item  \textbf{Cilj:} \textit{Upload} nalaza dobivenog pri privatnom pregledu na sustav
						\item  \textbf{Sudionici:} Baza podataka
						\item  \textbf{Preduvjet:} Odabir profila djeteta ili roditelja
						\item  \textbf{Opis osnovnog tijeka:}
						
						\item[] \begin{packed_enum}
							
							\item Na izborniku s lijeve strane sučelja roditelj odabire opciju "Učitavanje nalaza"
							\item Otvara se novo sučelje s desne strane izbornika, na kojem se nalazi polje za poruku liječniku/pedijatru, gumb za odabir dokumenta, te gumb "Učitaj"
							\item Roditelj piše poruku, odabire dokument i stišće gumb "Učitaj", čime se vrši \textit{upload} dokumenta i poruke
							\item Roditelj dobiva vizualnu potvrdu da je učitavanje uspjelo
						\end{packed_enum}
						
						\item  \textbf{Opis mogućih odstupanja:}
						
						\item[] \begin{packed_item}
							\item[3.a] Neuspjeh učitavanja dokumenta, zbog krivog formata dokumenta
							\item[] \begin{packed_enum}
								\item U slučaju pogrešnog formata dokumenta, ispisati će se upozorenje korisniku
								\item Korisnik pokušava učitati drugu vrstu dokumenta
							\end{packed_enum}
							\item[3.b] Neuspjeh učitavanja dokumenta, zbog greške u pristupu serverima
							\item[] \begin{packed_enum}
								\item U slučaju pogreške u mreži, ispisuje se poruka korisniku da provjeri svoju mrežnu povezanost
							\end{packed_enum}
							
						\end{packed_item}
					\end{packed_item}
					
					\noindent \underbar{\textbf{UC$<$broj obrasca$>$ -$<$ime obrasca$>$}}
					\begin{packed_item}
						
						\item \textbf{Glavni sudionik: }$<$sudionik$>$
						\item  \textbf{Cilj:} $<$cilj$>$
						\item  \textbf{Sudionici:} $<$sudionici$>$
						\item  \textbf{Preduvjet:} $<$preduvjet$>$
						\item  \textbf{Opis osnovnog tijeka:}
						
						\item[] \begin{packed_enum}
							
							\item $<$opis korak jedan$>$
							\item $<$opis korak dva$>$
							\item $<$opis korak tri$>$
							\item $<$opis korak četiri$>$
							\item $<$opis korak pet$>$
						\end{packed_enum}
						
						\item  \textbf{Opis mogućih odstupanja:}
						
						\item[] \begin{packed_item}
							
							\item[2.a] $<$opis mogućeg scenarija odstupanja u koraku 2$>$
							\item[] \begin{packed_enum}
								
								\item $<$opis rješenja mogućeg scenarija korak 1$>$
								\item $<$opis rješenja mogućeg scenarija korak 2$>$
								
							\end{packed_enum}
							\item[2.b] $<$opis mogućeg scenarija odstupanja u koraku 2$>$
							\item[3.a] $<$opis mogućeg scenarija odstupanja  u koraku 3$>$
							
						\end{packed_item}
					\end{packed_item}
					
					\noindent \underbar{\textbf{UC$<$broj obrasca$>$ -$<$ime obrasca$>$}}
					\begin{packed_item}
						
						\item \textbf{Glavni sudionik: }$<$sudionik$>$
						\item  \textbf{Cilj:} $<$cilj$>$
						\item  \textbf{Sudionici:} $<$sudionici$>$
						\item  \textbf{Preduvjet:} $<$preduvjet$>$
						\item  \textbf{Opis osnovnog tijeka:}
						
						\item[] \begin{packed_enum}
							
							\item $<$opis korak jedan$>$
							\item $<$opis korak dva$>$
							\item $<$opis korak tri$>$
							\item $<$opis korak četiri$>$
							\item $<$opis korak pet$>$
						\end{packed_enum}
						
						\item  \textbf{Opis mogućih odstupanja:}
						
						\item[] \begin{packed_item}
							
							\item[2.a] $<$opis mogućeg scenarija odstupanja u koraku 2$>$
							\item[] \begin{packed_enum}
								
								\item $<$opis rješenja mogućeg scenarija korak 1$>$
								\item $<$opis rješenja mogućeg scenarija korak 2$>$
								
							\end{packed_enum}
							\item[2.b] $<$opis mogućeg scenarija odstupanja u koraku 2$>$
							\item[3.a] $<$opis mogućeg scenarija odstupanja  u koraku 3$>$
							
						\end{packed_item}
					\end{packed_item}
					
					\noindent \underbar{\textbf{UC$<$broj obrasca$>$ -$<$ime obrasca$>$}}
					\begin{packed_item}
						
						\item \textbf{Glavni sudionik: }$<$sudionik$>$
						\item  \textbf{Cilj:} $<$cilj$>$
						\item  \textbf{Sudionici:} $<$sudionici$>$
						\item  \textbf{Preduvjet:} $<$preduvjet$>$
						\item  \textbf{Opis osnovnog tijeka:}
						
						\item[] \begin{packed_enum}
							
							\item $<$opis korak jedan$>$
							\item $<$opis korak dva$>$
							\item $<$opis korak tri$>$
							\item $<$opis korak četiri$>$
							\item $<$opis korak pet$>$
						\end{packed_enum}
						
						\item  \textbf{Opis mogućih odstupanja:}
						
						\item[] \begin{packed_item}
							
							\item[2.a] $<$opis mogućeg scenarija odstupanja u koraku 2$>$
							\item[] \begin{packed_enum}
								
								\item $<$opis rješenja mogućeg scenarija korak 1$>$
								\item $<$opis rješenja mogućeg scenarija korak 2$>$
								
							\end{packed_enum}
							\item[2.b] $<$opis mogućeg scenarija odstupanja u koraku 2$>$
							\item[3.a] $<$opis mogućeg scenarija odstupanja  u koraku 3$>$
							
						\end{packed_item}
					\end{packed_item}
					
					\noindent \underbar{\textbf{UC$<$broj obrasca$>$ -$<$ime obrasca$>$}}
					\begin{packed_item}
						
						\item \textbf{Glavni sudionik: }$<$sudionik$>$
						\item  \textbf{Cilj:} $<$cilj$>$
						\item  \textbf{Sudionici:} $<$sudionici$>$
						\item  \textbf{Preduvjet:} $<$preduvjet$>$
						\item  \textbf{Opis osnovnog tijeka:}
						
						\item[] \begin{packed_enum}
							
							\item $<$opis korak jedan$>$
							\item $<$opis korak dva$>$
							\item $<$opis korak tri$>$
							\item $<$opis korak četiri$>$
							\item $<$opis korak pet$>$
						\end{packed_enum}
						
						\item  \textbf{Opis mogućih odstupanja:}
						
						\item[] \begin{packed_item}
							
							\item[2.a] $<$opis mogućeg scenarija odstupanja u koraku 2$>$
							\item[] \begin{packed_enum}
								
								\item $<$opis rješenja mogućeg scenarija korak 1$>$
								\item $<$opis rješenja mogućeg scenarija korak 2$>$
								
							\end{packed_enum}
							\item[2.b] $<$opis mogućeg scenarija odstupanja u koraku 2$>$
							\item[3.a] $<$opis mogućeg scenarija odstupanja  u koraku 3$>$
							
						\end{packed_item}
					\end{packed_item}
					
					\noindent \underbar{\textbf{UC$<$broj obrasca$>$ -$<$ime obrasca$>$}}
					\begin{packed_item}
						
						\item \textbf{Glavni sudionik: }$<$sudionik$>$
						\item  \textbf{Cilj:} $<$cilj$>$
						\item  \textbf{Sudionici:} $<$sudionici$>$
						\item  \textbf{Preduvjet:} $<$preduvjet$>$
						\item  \textbf{Opis osnovnog tijeka:}
						
						\item[] \begin{packed_enum}
							
							\item $<$opis korak jedan$>$
							\item $<$opis korak dva$>$
							\item $<$opis korak tri$>$
							\item $<$opis korak četiri$>$
							\item $<$opis korak pet$>$
						\end{packed_enum}
						
						\item  \textbf{Opis mogućih odstupanja:}
						
						\item[] \begin{packed_item}
							
							\item[2.a] $<$opis mogućeg scenarija odstupanja u koraku 2$>$
							\item[] \begin{packed_enum}
								
								\item $<$opis rješenja mogućeg scenarija korak 1$>$
								\item $<$opis rješenja mogućeg scenarija korak 2$>$
								
							\end{packed_enum}
							\item[2.b] $<$opis mogućeg scenarija odstupanja u koraku 2$>$
							\item[3.a] $<$opis mogućeg scenarija odstupanja  u koraku 3$>$
							
						\end{packed_item}
					\end{packed_item}
					
					\noindent \underbar{\textbf{UC$<$broj obrasca$>$ -$<$ime obrasca$>$}}
					\begin{packed_item}
						
						\item \textbf{Glavni sudionik: }$<$sudionik$>$
						\item  \textbf{Cilj:} $<$cilj$>$
						\item  \textbf{Sudionici:} $<$sudionici$>$
						\item  \textbf{Preduvjet:} $<$preduvjet$>$
						\item  \textbf{Opis osnovnog tijeka:}
						
						\item[] \begin{packed_enum}
							
							\item $<$opis korak jedan$>$
							\item $<$opis korak dva$>$
							\item $<$opis korak tri$>$
							\item $<$opis korak četiri$>$
							\item $<$opis korak pet$>$
						\end{packed_enum}
						
						\item  \textbf{Opis mogućih odstupanja:}
						
						\item[] \begin{packed_item}
							
							\item[2.a] $<$opis mogućeg scenarija odstupanja u koraku 2$>$
							\item[] \begin{packed_enum}
								
								\item $<$opis rješenja mogućeg scenarija korak 1$>$
								\item $<$opis rješenja mogućeg scenarija korak 2$>$
								
							\end{packed_enum}
							\item[2.b] $<$opis mogućeg scenarija odstupanja u koraku 2$>$
							\item[3.a] $<$opis mogućeg scenarija odstupanja  u koraku 3$>$
							
						\end{packed_item}
					\end{packed_item}
					
					\noindent \underbar{\textbf{UC$<$broj obrasca$>$ -$<$ime obrasca$>$}}
					\begin{packed_item}
						
						\item \textbf{Glavni sudionik: }$<$sudionik$>$
						\item  \textbf{Cilj:} $<$cilj$>$
						\item  \textbf{Sudionici:} $<$sudionici$>$
						\item  \textbf{Preduvjet:} $<$preduvjet$>$
						\item  \textbf{Opis osnovnog tijeka:}
						
						\item[] \begin{packed_enum}
							
							\item $<$opis korak jedan$>$
							\item $<$opis korak dva$>$
							\item $<$opis korak tri$>$
							\item $<$opis korak četiri$>$
							\item $<$opis korak pet$>$
						\end{packed_enum}
						
						\item  \textbf{Opis mogućih odstupanja:}
						
						\item[] \begin{packed_item}
							
							\item[2.a] $<$opis mogućeg scenarija odstupanja u koraku 2$>$
							\item[] \begin{packed_enum}
								
								\item $<$opis rješenja mogućeg scenarija korak 1$>$
								\item $<$opis rješenja mogućeg scenarija korak 2$>$
								
							\end{packed_enum}
							\item[2.b] $<$opis mogućeg scenarija odstupanja u koraku 2$>$
							\item[3.a] $<$opis mogućeg scenarija odstupanja  u koraku 3$>$
							
						\end{packed_item}
					\end{packed_item}
				
					
				\subsubsection{Dijagrami obrazaca uporabe}
					
					\textit{Prikazati odnos aktora i obrazaca uporabe odgovarajućim UML dijagramom. Nije nužno nacrtati sve na jednom dijagramu. Modelirati po razinama apstrakcije i skupovima srodnih funkcionalnosti.}
				\eject		
				
			\subsection{Sekvencijski dijagrami}
				
				\textbf{\textit{dio 1. revizije}}\\
				
				\textit{Nacrtati sekvencijske dijagrame koji modeliraju najvažnije dijelove sustava (max. 4 dijagrama). Ukoliko postoji nedoumica oko odabira, razjasniti s asistentom. Uz svaki dijagram napisati detaljni opis dijagrama.}
				\eject
	
		\section{Ostali zahtjevi}
		
			\textbf{\textit{dio 1. revizije}}\\
		 
			 \textit{Nefunkcionalni zahtjevi i zahtjevi domene primjene dopunjuju funkcionalne zahtjeve. Oni opisuju \textbf{kako se sustav treba ponašati} i koja \textbf{ograničenja} treba poštivati (performanse, korisničko iskustvo, pouzdanost, standardi kvalitete, sigurnost...). Primjeri takvih zahtjeva u Vašem projektu mogu biti: podržani jezici korisničkog sučelja, vrijeme odziva, najveći mogući podržani broj korisnika, podržane web/mobilne platforme, razina zaštite (protokoli komunikacije, kriptiranje...)... Svaki takav zahtjev potrebno je navesti u jednoj ili dvije rečenice.}
			 
			 
			 
	