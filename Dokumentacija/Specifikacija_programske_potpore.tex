\chapter{Specifikacija programske potpore}
		
	\section{Funkcionalni zahtjevi}
			\noindent \textbf{Dionici:}
			
			\begin{packed_enum}
				
				\item Vlasnik (naručitelj)
				\item Roditelji				
				\item Liječnici i pedijatri
				\item Administrator
				\item Razvojni tim
				
			\end{packed_enum}
			
			\noindent \textbf{Aktori i njihovi funkcionalni zahtjevi:}
			
			
			\begin{packed_enum}
				\item  \underbar{Neregistrirani/neprijavljeni korisnik (inicijator) može:}
				
				\begin{packed_enum}
					
					\item pregledati katalog usluga aplikacije
					\item vidjeti koje sve vrste korisnika aplikacija podržava
					\item registrirati se u sustav, koristeći korisničko ime, email adresu, lozinku, ime, prezime i OIB, čime stvara osobni korisnički račun
					\item prijaviti se u sustav putem korisničkog imena i lozinke
					
				\end{packed_enum}
			
				\item  \underbar{Roditelj (inicijator) može:}
				
				\begin{packed_enum}
					
					\item pristupiti profilima svoje djece ili svojem profilu
					\item otvarati i čitati obavijesti vezane uz svaki profil, poslane od liječnika ili pedijatra
					\item vidjeti svoj ili djetetov medicinski karton te povijest pregleda
					\item pristupiti ispričnicama generiranim za pojedino dijete te potvrdama o bolovanju
					\item pristupiti naknadnim nalazima laboratorijskih pretraga
					\item primiti informacije i narudžbe na specijalističke preglede, te vidjeti lokacije na kojima je moguće izvesti navedeni pregled
					\item učitati (\textit{uploadati}) nalaze koje dobije prilikom pregleda u privatnoj ordinaciji
					
				\end{packed_enum}
				
				\item  \underbar{Pedijatar (inicijator) može:}
				
				\begin{packed_enum}
					
					\item pristupiti profilima djece kojima je dedicirani pedijatar
					\item pristupiti liječničkim kartonima svih svojih pacijenata
					\item prijaviti novo dijete prilikom prvog pregleda, koristeći osobne podatke djeteta (ime, prezime, OIB, datum rođenja), te OIB roditelja
					\item za svakog pacijenta unijeti novi pregled
					\item za svakog pacijenta generirati ispričnicu
					\item za roditelje djece preporučiti bolovanje
					\item naručiti dijete na specijalistički pregled, te preporučiti lokacije za izvedbu istog
					
				\end{packed_enum}
				
				\item  \underbar{Liječnik obiteljske medicine(inicijator) može:}
				
				\begin{packed_enum}
					
					\item pristupiti profilima roditelja kojima je liječnik
					\item pristupiti liječničkim kartonima svih svojih pacijenata
					\item za svakog pacijenta unijeti novi pregled
					\item potvrditi ili odbiti preporuke za bolovanje roditelja
					\item naručiti roditelja na specijalistički pregled, te preporučiti lokacije za izvedbu istog
					
				\end{packed_enum}
				
				\item  \underbar{Administrator(inicijator) može:}
				
				\begin{packed_enum}
					
					\item vidjeti popis svih registriranih korisnika i njihovih osobnih podataka
					\item brisati korisnike
					\item mijenjati veze između korisnika, npr. premjestiti dijete s profila jednog roditelja na profil drugog
					\item stvarati profile liječnika i pedijatara
					
				\end{packed_enum}
				
				\item  \underbar{Baza podataka(sudionik):}
				
				\begin{packed_enum}
					
					\item pohranjuje sve podatke o korisnicima, njihove međusobne povezanosti i uloge
					\item pohranjuje liječničke kartone i dijagnoze roditelja i djece
					
				\end{packed_enum}
				
			\end{packed_enum}
			
			\eject 
			
			
				
			\subsection{Obrasci uporabe}
				
				\textbf{\textit{dio 1. revizije}}
				
				\subsubsection{Opis obrazaca uporabe}
					\textit{Funkcionalne zahtjeve razraditi u obliku obrazaca uporabe. Svaki obrazac je potrebno razraditi prema donjem predlošku. Ukoliko u nekom koraku može doći do odstupanja, potrebno je to odstupanje opisati i po mogućnosti ponuditi rješenje kojim bi se tijek obrasca vratio na osnovni tijek.}\\
					

					\noindent \underbar{\textbf{UC$<$broj obrasca$>$ -$<$ime obrasca$>$}}
					\begin{packed_item}
	
						\item \textbf{Glavni sudionik: }$<$sudionik$>$
						\item  \textbf{Cilj:} $<$cilj$>$
						\item  \textbf{Sudionici:} $<$sudionici$>$
						\item  \textbf{Preduvjet:} $<$preduvjet$>$
						\item  \textbf{Opis osnovnog tijeka:}
						
						\item[] \begin{packed_enum}
	
							\item $<$opis korak jedan$>$
							\item $<$opis korak dva$>$
							\item $<$opis korak tri$>$
							\item $<$opis korak četiri$>$
							\item $<$opis korak pet$>$
						\end{packed_enum}
						
						\item  \textbf{Opis mogućih odstupanja:}
						
						\item[] \begin{packed_item}
	
							\item[2.a] $<$opis mogućeg scenarija odstupanja u koraku 2$>$
							\item[] \begin{packed_enum}
								
								\item $<$opis rješenja mogućeg scenarija korak 1$>$
								\item $<$opis rješenja mogućeg scenarija korak 2$>$
								
							\end{packed_enum}
							\item[2.b] $<$opis mogućeg scenarija odstupanja u koraku 2$>$
							\item[3.a] $<$opis mogućeg scenarija odstupanja  u koraku 3$>$
							
						\end{packed_item}
					\end{packed_item}
				
					
				\subsubsection{Dijagrami obrazaca uporabe}
					
					\textit{Prikazati odnos aktora i obrazaca uporabe odgovarajućim UML dijagramom. Nije nužno nacrtati sve na jednom dijagramu. Modelirati po razinama apstrakcije i skupovima srodnih funkcionalnosti.}
				\eject		
				
			\subsection{Sekvencijski dijagrami}
				
				\textbf{\textit{dio 1. revizije}}\\
				
				\textit{Nacrtati sekvencijske dijagrame koji modeliraju najvažnije dijelove sustava (max. 4 dijagrama). Ukoliko postoji nedoumica oko odabira, razjasniti s asistentom. Uz svaki dijagram napisati detaljni opis dijagrama.}
				\eject
	
		\section{Ostali zahtjevi}
		
			\textbf{\textit{dio 1. revizije}}\\
		 
			 \textit{Nefunkcionalni zahtjevi i zahtjevi domene primjene dopunjuju funkcionalne zahtjeve. Oni opisuju \textbf{kako se sustav treba ponašati} i koja \textbf{ograničenja} treba poštivati (performanse, korisničko iskustvo, pouzdanost, standardi kvalitete, sigurnost...). Primjeri takvih zahtjeva u Vašem projektu mogu biti: podržani jezici korisničkog sučelja, vrijeme odziva, najveći mogući podržani broj korisnika, podržane web/mobilne platforme, razina zaštite (protokoli komunikacije, kriptiranje...)... Svaki takav zahtjev potrebno je navesti u jednoj ili dvije rečenice.}
			 
			 
			 
	