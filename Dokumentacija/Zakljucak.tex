\chapter{Zaključak i budući rad}
		
		\text Zadatak našeg tima na ovome projektu bio je razvoj web aplikacije naziva "Ozdravi". Glavna ideja iza aplikacije jest olakšanje komunikacije između roditelja djece, pedijatara, liječnika obiteljske medicine, škola i poslodavaca. Razvoj aplikacije trajao je 13 tjedana, za vrijeme kojih se dinamika tima i njegova efikasnost drastično mijenjala. Ipak, na kraju je postignut cilj projekta, razvijena je aplikacija. Podijelio bih rad na projektu u tri faze.\\
		Prva faza sastojala se od početne organizacije te nekoliko uzastopnih sastanaka na kojima je razvijana ideja i arhitektura aplikacije, te na kojima su se članovi tima međusobno upoznali. U ovoj fazi izglasan je i voditelj tima, čiju ulogu preuzeo je Jan Komerički. Također, u ovoj fazi određena su 4 pod-tima, svaki sa svojim zadacima. Luka Žaja i Lovro Matić oformili su tim za rad s bazom podataka, koji se u kasnijoj fazi projekta stopio u tim za razvoj \textit{backend} potpore. Taj tim oformili su Dino Dubinović i Luka Bračun. Treći tim sastojao se od Kristine Čavlović i Ante Prolića, koji su odlučili raditi na razvoju potpore za \textit{frontend} dio aplikacije. Zadnji, jednočlani tim, sastojao se od Jana Komeričkog, čiji je zadatak postao pisanje kompletne dokumentacije projekta i koordinacija rada ostala tri tima. \\
		Druga faza rada tima bio je razvoj inicijalne verzije aplikacije, a krajnji rok ove faze bio je 17.11.2023. U ovoj fazi svi su se članovi tima upoznali s tehnologijama koje su bile potrebne za razvoj aplikacije. Fazu okarakterizirao je spor i metodičan rad tima, dijelom zbog još nezrele suradnje među članovima, a dijelom zbog manjka znanja o tehnologijama. Velika većina dokumentacije također je završena u ovoj fazi, te većina UML dijagrama koji su potrebni za prikaz rada aplikacije. Ipak, faza je završila vrlo uspješno, s vrlo pozitivnim povratnim informacijama dionika.\\
		Treća faza rada sastojala se od raspodijele preostalog posla i nastavka rada na razvoju aplikacije. Zadnji tjedan rada bio je pomalo kaotičan zbog ponekad nejasne komunikacije između pod-timova, no projekt i dokumentacija uspješno su završeni.\\
		Ovaj projekt bio je prvi ozbiljni i opsežni projekt razvoja programske potpore za većinu nas. Upoznali smo nove tehnologije, pogotovo \textit{Spring Boot} i \textit{React}, te smo naučili o važnosti dokumentacije. Kvalitetna dokumentacija jedan je od najkorisnijih pomoćnika pri razvoju, jer se po njoj mogu orijentirati svi članovi tima. Sve u svemu, naučili smo kako raditi i komunicirati u timu, te kako se koristiti modernim tehnologijama za programske potpore. Mislim da svi odlazimo od ovog projekta iznimno zadovoljni postignutim, usprkos velikoj količini mogućih unaprijeđenja koja se mogu dodati aplikaciji u budućnosti. Određeni članovi tima čak su izrazili želju za daljnjim radom unutar ovog tima, ali na nekom vlastitom projektu, tako da je moguće da će \textit{Proggy i Žohari} nastaviti razvijati \textit{software}, bar još neko vrijeme. 
													:)
		
		\textit{Note: Admin tools nisu funkcionalni prilikom finalne predaje.}
		\eject 